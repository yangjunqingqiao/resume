%%%%%%%%%%%%%%%%%%%%%%%%%%%%%%%%%%%%%%%
% Wenneker Resume/CV
% LaTeX Template
% Version 1.1 (19/6/2016)
%
% This template has been downloaded from:
% http://www.LaTeXTemplates.com
%
% Original author:
% Frits Wenneker (http://www.howtotex.com) with extensive modifications by 
% Vel (vel@LaTeXTemplates.com)
%
% License:
% CC BY-NC-SA 3.0 (http://creativecommons.org/licenses/by-nc-sa/3.0/
%
%%%%%%%%%%%%%%%%%%%%%%%%%%%%%%%%%%%%%%

\documentclass[a4paper,12pt]{memoir} 
%%%%%%%%%%%%%%%%%%%%%%%%%%%%%%%%%%%%%%%%%
% Wenneker Resume/CV
% Structure Specification File
% Version 1.1 (19/6/2016)
%
% This file has been downloaded from:
% http://www.LaTeXTemplates.com
%
% Original author:
% Frits Wenneker (http://www.howtotex.com) with extensive modifications by 
% Vel (vel@latextemplates.com)
%
% License:
% CC BY-NC-SA 3.0 (http://creativecommons.org/licenses/by-nc-sa/3.0/)
%
%%%%%%%%%%%%%%%%%%%%%%%%%%%%%%%%%%%%%%%%%

%----------------------------------------------------------------------------------------
%	PACKAGES AND OTHER DOCUMENT CONFIGURATIONS
%----------------------------------------------------------------------------------------

\usepackage{XCharter} % Use the Bitstream Charter font
\usepackage[utf8]{inputenc} % Required for inputting international characters
\usepackage[T1]{fontenc} % Output font encoding for international characters

\usepackage[top=1cm,left=1cm,right=1cm,bottom=1cm]{geometry} % Modify margins

\usepackage{graphicx} % Required for figures

\usepackage{flowfram} % Required for the multi-column layout

\usepackage{url} % URLs
\usepackage{hyperref} % Hyperlinks

\usepackage[usenames,dvipsnames]{xcolor} % Required for custom colours

\usepackage{tikz} % Required for the horizontal rule

\usepackage{enumitem} % Required for modifying lists

\usepackage{hyperref}

\hypersetup{
    colorlinks=true,
    linkcolor=green,
    filecolor=magenta,
    urlcolor=blue,
}

\setlist{noitemsep,nolistsep} % Remove spacing within and around lists

\setlength{\columnsep}{\baselineskip} % Set the spacing between columns

% Define the left frame (sidebar)
\newflowframe{0.22\textwidth}{\textheight}{-10pt}{0pt}[left]
\newlength{\LeftMainSep}
\setlength{\LeftMainSep}{0.2\textwidth}
\addtolength{\LeftMainSep}{1\columnsep}
 
% Small static frame for the vertical line
\newstaticframe{1.5pt}{\textheight}{\LeftMainSep}{0pt}
 
% Content of the static frame with the vertical line
\begin{staticcontents}{1}
\hfill
\tikz{\draw[loosely dotted,color=ForestGreen,line width=1.5pt,yshift=0](0,0) -- (0,\textheight);}
\hfill\mbox{}
\end{staticcontents}
 
% Define the right frame (main body)
\addtolength{\LeftMainSep}{1.5pt}
\addtolength{\LeftMainSep}{1\columnsep}
\newflowframe{0.7\textwidth}{\textheight}{\LeftMainSep}{0pt}[main01]

\pagestyle{empty} % Disable all page numbering

\setlength{\parindent}{0pt} % Stop paragraph indentation

%----------------------------------------------------------------------------------------
%	NEW COMMANDS
%----------------------------------------------------------------------------------------

\newcommand{\userinformation}[1]{\renewcommand{\userinformation}{#1}} % Define a new command for the CV user's information that goes into the left column

\newcommand{\cvheading}[1]{{\Large\bfseries\color{ForestGreen} #1} \par\vspace{.6\baselineskip}} % New command for the CV heading
\newcommand{\cvsubheading}[1]{{\Large\bfseries #1} \bigbreak} % New command for the CV subheading

\newcommand{\Sep}{\vspace{1em}} % New command for the spacing between headings
\newcommand{\SmallSep}{\vspace{0.3em}} % New command for the spacing within headings

\newcommand{\aboutme}[2]{ % New command for the about me section
\textbf{\color{ForestGreen} #1}~~#2\par\Sep
}
	
\newcommand{\CVSection}[1]{ % New command for the headings within sections
{\Large\textbf{#1}}\par
\SmallSep % Used for spacing
}

\newcommand{\CVItem}[2]{ % New command for the item descriptions
\textbf{\color{ForestGreen} #1}\par
#2
\SmallSep % Used for spacing
}

\newcommand{\SidebarCategory}[1]{
    {\Large\textbf{#1}}
    \SmallSep
}

\newcommand{\SidebarItem}[2]{ % New command for the item descriptions
\textbf{\color{ForestGreen} #1}
#2
\SmallSep % Used for spacing
}

\newcommand{\bluebullet}{\textcolor{ForestGreen}{$\circ$}~~} % New command for the blue bullets

\newcommand{\smolbullet}{\textcolor{Black}~~~~~~{$\circ$}~~} % New command for the blue bullets


%----------------------------------------------------------------------------------------
%	NAME AND CONTACT INFORMATION 
%----------------------------------------------------------------------------------------

\userinformation{
	\raggedleft
		\small
		\textcolor{ForestGreen}{YangJunqing Qiao} \\ 
		\textit{junqingqiao@gmail.com} \\ 
		% \href{https://yangjunqingqiao.github.io/}{Personal Website} \\ 
		\href{https://www.linkedin.com/in/junqing-qiao/}{LinkedIn} \\
		(978) 905-9534 \\ 
		\sep
	\raggedright
}

%----------------------------------------------------------------------------------------

\begin{document}

\userinformation
\framebreak

%----------------------------------------------------------------------------------------
%	EXPERIENCE
%----------------------------------------------------------------------------------------

\CVSection{Experience}

\CVItem
	{\textit{Software Engineer}, Pison \hfill Aug 2021 - Present}
	{
		\begin{itemize}
			\item \textbf{Own}, \textbf{upgrade}, and \textbf{maintain} backend ML library that powers Pison's neural gesture inferences. \textit{(C++)}
			\item \textbf{Optimize} and \textbf{productionize algorithms} designed by the ML team and \textbf{designed APIs} used by the front-end team to support Pison's next gen neural inference pipeline. \textit{(C++, Python, Java)}
			\item \textbf{Parallelized} and \textbf{cached} feature extraction to \textbf{reduce computation time} by \textbf{40\%} during training and inference. \textit{(C++)}
			\item \textbf{Ported} Pison's native ML library to Apple's \textbf{M1 arm64} chips. \textit{(C++)}
		\end{itemize}
	}

\CVItem
	{\textit{Software Engineer II}, Raytheon \hfill Jun 2020 - Aug 2021}
	{
		\begin{itemize}
			\item \textbf{Maintained} and \textbf{upgraded} real time mission critical systems across 5 different products. \textit{(C++)}
			\item \textbf{Increased automated test coverage} for legacy and emerging products. \textit{(C++, Matlab, Java, Perl, Lua)}
			\item \textbf{Generalize} and \textbf{extended} mature product paradigms to newer products. \textit{(C++)}
		\end{itemize}
	}

\CVItem
	{\textit{Software Engineer Intern}, Raytheon \hfill Jun 2019 - Aug 2019}
	{
		\begin{itemize}
			\item \textbf{Decreased latency of operator commands} by parallelizing legacy status polling thread of radar operating software \textit{(C++)}
			\item \textbf{Streamlined software testing process} by developing a hardware emulator to simulate operator commands. \textit{(Python)}
			\item \textbf{Increased the validity of standard testing procedure} by helping to identify and resolve long standing segfault. \textit{(C++)}
		\end{itemize}
	}

\CVItem
	{\textit{Junior Investigator}, REUMass Amherst \hfill Jun 2017 - Aug 2017}
	{
		\begin{itemize}
			\item Survey paper detailing the \textbf{insights circuit and complexity theory have on the power and behavior of neural networks}
			\item Showed why \textbf{neural networks are hard to analyze} and understand on a theoretical level by compiling several theorems reducing them to \textbf{unsolved problems in circuit theory}.
		\end{itemize}
	}

\Sep

%----------------------------------------------------------------------------------------
%	PROJECTS
%----------------------------------------------------------------------------------------

% \CVSection{Projects}

% \CVItem
% 	{\textit{Detecting Latent Heuristics in BERT}, Reproducibility Paper \hfill 2019}
% 	{
% 		\begin{itemize}
% 			\item Tested \textbf{BERT} on two traditionally difficult NLP tasks to \textbf{probe} its \textbf{reliance on latent statistical heuristics and biases} in the datasets. \textit{(Python, Pytorch)}
% 		\end{itemize}
% 	}

% \CVItem
% 	{\textit{Gaussian Normalization}, Neural Networks Final Project \hfill 2018}
% 	{
% 		\begin{itemize}
% 			\item Extending the ideas of batch and copula normalization, I showed why \textbf{normalizing arbitrary layers of a neural network with a Gaussian distribution} is impractical. \textit{(Python, Pytorch)}
% 		\end{itemize}
% 	}
% \Sep

%----------------------------------------------------------------------------------------
%	SKILLS
%----------------------------------------------------------------------------------------

\CVSection{Skills}

\CVItem{Programming}
{
	\begin{tabular}{p{0.2\textwidth} p{0.2\textwidth} p{0.2\textwidth}}
		\bluebullet Python &  \bluebullet C++ & \bluebullet Java\\
		\bluebullet Shell &  \bluebullet SQL\\
	\end{tabular}
}

\Sep

%----------------------------------------------------------------------------------------
%	EDUCATION
%----------------------------------------------------------------------------------------

\CVSection{Education}

\CVItem{University of Massachusetts, Amherst}
{
	\begin{itemize}
		\item MS in Computer Science \hfill \textcolor{ForestGreen}{\textit{Sept 2018 - Feb 2020}} \\
		\item BS in Computer Science and Mathematics \hfill \textcolor{ForestGreen}{\textit{Sept 2014 - Jun 2018}}
	\end{itemize}
}

\Sep

%----------------------------------------------------------------------------------------
%	NEW PAGE DELIMITER
%	Place this block wherever you would like the content of your CV to go onto the next page
%----------------------------------------------------------------------------------------

% \clearpage % Start a new page

% \userinformation % Print your information in the left column

% \framebreak % End of the first column

%----------------------------------------------------------------------------------------
%	VOLUNTEER
%---------------------------------------------------------------------------------------
\CVSection{Volunteering}

\CVItem{CoronaWhy \hfill Apr 2020 - Present}{
	\begin{itemize}
		\item Developing a research tool to \textbf{parse contradictory claims} in \textbf{Covid-19 medical literature} to help \textbf{rank} the \textbf{effectiveness} of studied \textbf{vaccines} and \textbf{theraputics}.
		\item \textbf{Productionized research notebooks} using \textbf{CI} techniques. \textit{(Python)}
		\item \textbf{Integrated pretrained models} to research pipeline. \textit{(Python, Pytorch)}
	\end{itemize}
}
\Sep
%----------------------------------------------------------------------------------------
%	AWARDS
%----------------------------------------------------------------------------------------

\CVSection{Awards}

\CVItem
	{\textit{Datathon Finalist}, Brown University \hfill 2020}
	{
		\begin{itemize}
			\item Using \textbf{RNNs} to predict laser melt pool dimensions over time. \textit{(Python)}
	}

\end{document}
